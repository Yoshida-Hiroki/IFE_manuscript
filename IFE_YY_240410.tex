\documentclass[aps,prb,longbibliography,superscriptaddress,twocolumn]{revtex4-2}
\usepackage{graphicx,caption}
\usepackage{mathtools,amsmath,amssymb}
\usepackage{physics}
\usepackage{here}
\usepackage{multirow}
\usepackage{xcolor}
\usepackage{hyperref}

\def\be#1\ee{\begin{align}#1\end{align}}

%%%%%%%%%%%%%%%%%%%%%%%%%%%%%%%%%%%%%%%%%%%%%%%%%%%%%%%%%%%%%%%%%%%%%%%%%%%%%%

\begin{document}
\title{Inverse Faraday Effect Induced by the Quantum Geometry}

\author{Hiroki Yoshida}
\affiliation{Department of Physics, Institute of Science Tokyo, 2-12-1 Ookayama, Meguro-ku, Tokyo 152-8551, Japan}
\author{Takehito Yokoyama}
\affiliation{Department of Physics, Institute of Science Tokyo, 2-12-1 Ookayama, Meguro-ku, Tokyo 152-8551, Japan}


\date{\today}


\begin{abstract}
    We propose a novel mechanism of the inverse Faraday effect arising from a quantum geometric origin, characterized by the quantum metric dipole. Within a semiclassical framework based on the Boltzmann transport theory, we establish a general formalism describing light-induced magnetization in electronic systems as a second-order nonlinear response to the electric field of light. We show that this effect is activated by a spatially non-uniform electric field component and, in contrast to the conventional inverse Faraday effect, persists even under linearly polarized illumination. Model calculations reveal that the magnitude of the resulting magnetization is comparable to that induced in conventional inverse Faraday effect by circularly polarized light, indicating its experimental realizability. Our results highlight a direct manifestation of the quantum metric dipole in nonlinear optical responses and suggest a viable pathway for its experimental detection.
\end{abstract}

\maketitle

Faraday effect is a well-known magneto-optical effect... It's inverse effect, known as the inverse Faraday effect (IFE) is proposed and observed. For IFE, we need to use circularly polarized light to break the time-reversal symmetry of the system because the magnetization requires broken time-reversal symmetry.

Quantum geometry of Bloch electrons in materials gather attention to describe various phenomena such as non-linear Hall effect and ... Quantum metric, which measures quantum distance of quantum states, appear in the equation of motion of an electron under non-uniform electric field~\cite{Lapa_Hughes_2019}. 

In this paper, we reveal a new mechanism of light-induced magnetization in electronic systems. Application of non-uniform electric field alters the motion of electrons and it creates current, and thus magnetization. This magnetization, involve quantum metric effect and not limited to circularly polarized light.

\section{Derivation}
We start from the semiclassical equation of motion of an electron under non-uniform electric field~\cite{Lapa_Hughes_2019}. We expand the position-dependent electric field $\vb{E}(\vb{x})$ around origin as
\be
    E_{\mu}(\vb{x})&=e_{\mu}+e_{\mu\nu}x_{\nu},
\ee
where $\mu=x,y,z$ is the cartesian coordinate, $e_{\mu}=E_{\mu}(\vb{x}=\vb{0})$ and $e_{\mu\nu}=\partial E_{\mu}/\partial x_{\nu}\eval_{\vb{x}=\vb{0}}$. Then, in Ref.~\cite{Lapa_Hughes_2019}, the equation of motion (EOM) of an electron is derived as
\be
    \dot{x}_{\mu}&=\frac{1}{\hbar}\pdv{\varepsilon}{k_{\mu}}-\Omega_{\mu\nu}\dot{k}_{\nu}-\frac{1}{2\hbar}Qe_{\nu\lambda}\pdv{g_{\nu\lambda}}{k_{\mu}},\\
    \dot{k_{\mu}}&=\frac{1}{\hbar}QE_{\mu}(\vb{x}).
\ee
Here, $\varepsilon$ is energy of the electron, $\Omega_{\mu\nu}\coloneqq$ is the Berry curvature, $g_{\nu\lambda}\coloneqq$ is the quantum metric.

We first calculate electric current density of this system using the Boltzmann formalism. Boltzmann equation with relaxation time approximation can be written as~[reference]
\be
    \pdv{f}{t}+\dot{\vb{x}}\cdot\pdv{f}{\vb{x}}+\dot{\vb{k}}\cdot\pdv{f}{\vb{k}}=-\frac{f-f_0}{\tau},
\ee
where $f$ is the distribution function of the system, $f_0$ is the equilibrium Fermi distribution function, and $\tau$ is the relaxation time of an electron. We calculate the change of the distribution function $f-f_0$ up to a first order of $\tau$. In this case, $f$ in the left hand side can be approximated to be $f_0$ and $\pdv{f_0}{t}=0$ and $\pdv{f_0}{\vb{x}}=\vb{0}$. We get
\be
    f&\approx f_0+\tau \dot{k}_{\mu}\pdv{f_0}{k_{\mu}}\nonumber\\
    &=f_0+\frac{\tau Q}{\hbar}E_{\mu}(\vb{x})\pdv{f_0}{k_{\mu}}.
\ee
This distribution function has the same form as the one under uniform electric field. Using this distribution function, the electric current can be calculated as
\be
    j_{\mu}&\coloneqq Q\int[\mathrm{d}\vb{k}]f\dot{x}_{\mu}\nonumber\\
    &=Q\int[\mathrm{d}\vb{k}]\qty(f_0+\frac{\tau Q}{\hbar}E_{\mu}(\vb{x})\pdv{f_0}{k_{\mu}})\times\qty(\frac{1}{\hbar}\pdv{\varepsilon}{k_{\mu}}-\frac{Q}{\hbar}\Omega_{\mu\nu}E_{\nu}(\vb{x})-\frac{Q}{2\hbar}e_{\nu\lambda}\pdv{g_{\nu\lambda}}{k_{\mu}})\\
    &=\frac{Q}{\hbar}\int[\mathrm{d}\vb{k}]f_0\pdv{\varepsilon}{k_{\mu}}\nonumber\\
    &\quad+\qty(\frac{\tau Q^2}{\hbar^2}\int[\mathrm{d}\vb{k}]\pdv{f_0}{k_{\nu}}\pdv{\varepsilon}{k_{\mu}}-\frac{Q^2}{\hbar}\int[\mathrm{d}\vb{k}]f_0\Omega_{\mu\nu})E_{\nu}(\vb{x})-\frac{Q^2}{2\hbar}\int[\mathrm{d}\vb{k}]f_0\pdv{g_{\nu\lambda}}{k_{\mu}}e_{\nu\lambda}\nonumber\\
    &\qquad -\frac{\tau Q^3}{\hbar^2}\int[\mathrm{d}\vb{k}]\pdv{f_0}{k_{\nu}}\Omega_{\mu\gamma}E_{\nu}E_{\gamma}-\frac{\tau Q^3}{2\hbar^2}\int[\mathrm{d}\vb{k}]\pdv{f_0}{k_{\gamma}}\pdv{g_{\nu\lambda}}{k_{\mu}}E_{\gamma}e_{\nu\lambda}.\label{eq:current_general}
\ee
Magnetization and the current density are related by the Maxwell equations:
\be
    \vb{j}&=\curl\vb{H}-\pdv{\vb{D}}{t},
\ee
where
\be
    \vb{H}&=\frac{1}{\mu_0}\vb{B}-\vb{M},\\
    \vb{D}&=\varepsilon_0\vb{E}+\vb{P}.
\ee
Here, we only consider the case where the external magnetic field is zero $\vb{B}=\vb{0}$ and the system is in a steady state $\pdv{\vb{D}}{t}=0$. Thus,
\be
    \vb{j}&=-\curl{\vb{M}}.
\ee
We focus on a magnetization that is induced as a second-order response to the electric field of light and assume that
\be
    M_{\mu}&=a_{\mu\alpha\beta}E_{\alpha}(\vb{x})E_{\beta}(\vb{x}).
\ee
Then, the current can be calculated as
\be
    j_a &= -\varepsilon_{abc}\partial_{b}M_{c}\nonumber\\
    &= -\varepsilon_{abc}\partial_{b}\qty(a_{c\alpha\beta}E_{\alpha}(\vb{x})E_{\beta}(\vb{x}))\nonumber\\
    &=-2\varepsilon_{abc}a_{c\alpha\beta}E_{\alpha}(\vb{x})\partial_bE_{\beta}(\vb{x})\nonumber\\
    &\approx -2\varepsilon_{abc}a_{c\alpha\beta}E_{\alpha}(\vb{x})e_{\beta b}
\ee
By comparing this expression with Eq.~\eqref{eq:current_general}, we find that the last term of Eq.~\eqref{eq:current_general} is responsible for the current caused by the magnetization. We get
\be
    \varepsilon_{\mu\lambda c}a_{c\gamma\nu}&=\frac{\tau Q^3}{4\hbar^2}\int[\mathrm{d}\vb{k}]\pdv{f_0}{k_{\gamma}}\pdv{g_{\nu\lambda}}{k_{\mu}}E_{\gamma}e_{\nu\lambda}.
\ee
Thus, we get a formula for the coefficient
\be
    a_{\mu\alpha\beta}&=\frac{\tau Q^3}{\hbar^2}\varepsilon_{\mu\nu\delta}\int[\mathrm{d}\vb{k}]\pdv{f_0}{k_{\alpha}}\pdv{g_{\delta\beta}}{k_{\nu}}.\label{eq:coeff_metric}.
\ee
From quantum metric, Christoffel symbol can be defined as
\be
    \Gamma_{\alpha\beta\gamma}&\coloneqq\frac{1}{2}\qty(\partial_{k_{\gamma}}g_{\alpha\beta}+\partial_{k_{\beta}}g_{\alpha\gamma}-\partial_{k_{\alpha}}g_{\beta\gamma}).
\ee
Then, it is easy to see
\be
    \partial_{k_{\gamma}}g_{\alpha\beta}&=\Gamma_{\alpha\beta\gamma}+\Gamma_{\beta\alpha\gamma}.
\ee
Using this relation, the coefficient can also be written as
\be
    a_{\mu\alpha\beta}&=\frac{\tau Q^3}{\hbar^2}\varepsilon_{\mu\nu\delta}\int[\mathrm{d}\vb{k}]\pdv{f_0}{k_{\alpha}}\qty(\Gamma_{\delta\beta\nu}+\Gamma_{\beta\delta\nu}).\label{eq:coeff_chris}
\ee

\section{Model Calculation}
Two-dimensional massive Dirac Hamiltonian
\be
    H(k_x,k_y)&=k_x\sigma_x+k_y\sigma_y+M\sigma_z
\ee



\bibliography{IFE.bib}

\end{document}