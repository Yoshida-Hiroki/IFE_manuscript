\documentclass[aps,prb,longbibliography,superscriptaddress,twocolumn]{revtex4-2}
\usepackage{graphicx,caption}
\usepackage{mathtools,amsmath,amssymb}
\usepackage{physics}
\usepackage{here}
\usepackage{multirow}
\usepackage{xcolor}
\usepackage{hyperref}

\def\be#1\ee{\begin{align}#1\end{align}}

%%%%%%%%%%%%%%%%%%%%%%%%%%%%%%%%%%%%%%%%%%%%%%%%%%%%%%%%%%%%%%%%%%%%%%%%%%%%%%

\begin{document}
\title{Inverse Faraday Effect Induced by the Quantum Geometry}

\author{Hiroki Yoshida}
\affiliation{Department of Physics, Institute of Science Tokyo, 2-12-1 Ookayama, Meguro-ku, Tokyo 152-8551, Japan}
\author{Takehito Yokoyama}
\affiliation{Department of Physics, Institute of Science Tokyo, 2-12-1 Ookayama, Meguro-ku, Tokyo 152-8551, Japan}


\date{\today}


\begin{abstract}
    We propose a novel mechanism of the inverse Faraday effect arising from a quantum geometric origin, characterized by the quantum metric dipole. Within a semiclassical framework based on the Boltzmann transport theory, we establish a general formalism describing light-induced magnetization in electronic systems as a second-order nonlinear response to the electric field of light. We show that this effect is activated by a spatially non-uniform electric field component and, in contrast to the conventional inverse Faraday effect, persists even under linearly polarized illumination. Model calculations reveal that the magnitude of the resulting magnetization is comparable to that induced in conventional inverse Faraday effect by circularly polarized light, indicating its experimental realizability. Our results highlight a direct manifestation of the quantum metric dipole in nonlinear optical responses and suggest a viable pathway for its experimental detection.
\end{abstract}

\maketitle

The Faraday effect is one of the most prominent magneto-optical phenomena, characterized by the rotation of the polarization plane of light as it propagates through a magnetized medium~\cite{Faraday_1846}. Its counterpart, the inverse Faraday effect (IFE), refers to the generation of magnetization in a material in response to light. The IFE has been both theoretically predicted~\cite{Pitaevskii_1961,Pershan_1963} and experimentally confirmed~\cite{Zeil_Optically_1965}, and is typically described as a second-order optical response, with the induced magnetization being proportional to the cross product of the electric field and its complex conjugate. As a consequence, the IFE is conventionally associated with circularly polarized light. Both the Faraday and inverse Faraday effects now play central roles in the optical manipulation and probing of magnetization in condensed matter systems.

Nonlinear optical effects--which arise from higher-order responses (second order or above) to the electric and magnetic fields of light--have garnered growing theoretical interest in recent years. This attention is largely driven by their profound connection to quantum geometry. A paradigmatic example is the bulk photovoltaic effect~\cite{kraut.vonbaltz1979a,belinicher.sturman1980,vonbaltz.kraut1981a,aversa.sipe1995a,sipe.shkrebtii2000,fridkin2001}, a second-order response to the electric field, where the nonlinear conductivity can be expressed in terms of geometric quantities such as the Berry curvature~\cite{morimoto.nagaosa2016,morimoto.nagaosa2016b,ma.etal2021,ahn.etal2022}. Other nonlinear transport phenomena, such as the nonlinear Hall effect is known to be related with the Berry curvature dipole~\cite{Fu_2015}. Also, different geometric quantity known as a quantum metric dipole can appear in nonlinear Hall effect~\cite{Gao_positional_2014,Liu_2021,Kamal_2023,Gao_Science_2023}. Nevertheless, the influence of quantum geometry on the inverse Faraday effect remains largely unexplored.

In this letter, we unveil a novel mechanism of light-induced magnetization in electronic systems, mediated by quantum geometric effects. We consider spatially non-uniform light, characterized by an electric field with explicit position dependence. This spatial variation modifies the semiclassical equation of motion for electrons~\cite{Lapa_Hughes_2019}, which in turn alters the second-order current response derived from the Boltzmann transport equation. The resulting current, arising from the spatial inhomogeneity of the electric field, can be interpreted as a magnetization current, allowing us to derive an expression for light-induced magnetization as a second-order response to the electric field. This effect originates from Fermi surface contributions and is therefore relevant only in metallic systems. Remarkably, unlike the conventional inverse Faraday effect, it can be induced even by linearly polarized light. We further demonstrate the magnitude of this effect through a model calculation based on a simple Dirac Hamiltonian.

We start from the semiclassical equation of motion of an electron under non-uniform electric field. Here we consider time-independent electric field \textcolor{blue}{under an assumption that the time variation of an electric field is slow enough}. Then, our setup is the same as in Ref.~\cite{Lapa_Hughes_2019}. The Hamiltonian of the system is given as
\be
    \hat{H}&=\hat{H}_0+Q\phi(\hat{\vb{x}}),
\ee
where $H_0$ is a static Hamiltonian including a periodic potential,$Q=-e$ is the charge of an electron, and $\phi$ is a scalar potential that gives the electric field $E_{\mu}=-\pdv{\phi}{x_{\mu}}\ (\mu=x,y,z)$. We expand the position-dependent electric field $\vb{E}(\vb{x})$ around the origin as 
\be
    E_{\mu}(\vb{x})&=e_{\mu}+e_{\mu\nu}x_{\nu},
\ee
where $e_{\mu}=E_{\mu}(\vb{x}=\vb{0})$ and $e_{\mu\nu}=\pdv{E_{\mu}}{x_{\nu}}\eval_{\vb{x}=\vb{0}}$ are the electric field and its gradient at the origin, respectively. Because the electric field is a gradient of the scalar potential, $e_{\mu\nu}=e_{\nu\mu}$. In Ref.~\cite{Lapa_Hughes_2019}, the equation of motion (EOM) of an electron in this setting is derived as
\be
    \dot{x}_{\mu}&=\frac{1}{\hbar}\pdv{\varepsilon}{k_{\mu}}-\Omega_{\mu\nu}\dot{k}_{\nu}-\frac{1}{2\hbar}Qe_{\nu\lambda}\pdv{g_{\nu\lambda}}{k_{\mu}},\label{eq:EOM_x}\\
    \dot{k_{\mu}}&=\frac{1}{\hbar}QE_{\mu}(\vb{x})\label{eq:EOM_k}.
\ee
Here, $\varepsilon$ is the energy of the electron satisfying $\hat{H}_0\ket{\phi(\vb{k})}=\varepsilon(\vb{k})\ket{\phi(\vb{k})}$, focusing on the single-band Bloch state and $\Omega_{\mu\nu}\coloneqq i\qty(\bra{\pdv{u}{k_{\mu}}}\ket{\pdv{u}{k_{\nu}}}-\qty(\mu\leftrightarrow\nu))$ is the Berry curvature, and
\be
    g_{\mu\nu}\coloneqq \frac{1}{2}\qty(\bra{\pdv{u}{k_{\mu}}}\ket{\pdv{u}{k_{\nu}}}-\bra{\pdv{u}{k_{\mu}}}\ket{u}\bra{u}\ket{\pdv{u}{k_{\nu}}}+\qty(\mu\leftrightarrow\nu))
\ee
is the quantum metric. These quantities correspond to the imaginary and real part of the quantum geometric tensor~\cite{Provost_1980}, respectively. In the equations of motion, the first term of Eq.~\eqref{eq:EOM_x} is the conventional term showing the group velocity. The second term of Eq.~\eqref{eq:EOM_x} and the other Eq.~\eqref{eq:EOM_k} include the effect of an applied electric field. The third term of Eq.~\eqref{eq:EOM_x} is unique to the non-uniform electric field.

From the EOMs, we next calculate the electric current density of this system using the Boltzmann formalism. Boltzmann equation with relaxation time approximation can be written as
\be
    \pdv{f}{t}+\dot{\vb{x}}\cdot\pdv{f}{\vb{x}}+\dot{\vb{k}}\cdot\pdv{f}{\vb{k}}=-\frac{f-f_0}{\tau},
\ee
where $f$ is the distribution function of the system, $f_0$ is the equilibrium Fermi distribution function, and $\tau$ is the relaxation time of an electron. We calculate the change of the distribution function $f-f_0$ up to a first order of $\tau$. In this case, $f$ in the left hand side can be approximated to be $f_0$. Suppose that $\pdv{f_0}{t}=0$ and $\pdv{f_0}{\vb{x}}=\vb{0}$, we get
\be
    f&\approx f_0+\frac{\tau Q}{\hbar}E_{\mu}(\vb{x})\pdv{f_0}{k_{\mu}}.
\ee
This distribution function has the same form as the one under uniform electric field. Using this distribution function, the electric current can be calculated as
\begin{widetext}
    \be
        j_{\mu}&\coloneqq Q\int[\mathrm{d}\vb{k}]f\dot{x}_{\mu}\nonumber\\
        &=\frac{Q}{\hbar}\int[\mathrm{d}\vb{k}]f_0\pdv{\varepsilon}{k_{\mu}}+\qty(\frac{\tau Q^2}{\hbar^2}\int[\mathrm{d}\vb{k}]\pdv{f_0}{k_{\nu}}\pdv{\varepsilon}{k_{\mu}}-\frac{Q^2}{\hbar}\int[\mathrm{d}\vb{k}]f_0\Omega_{\mu\nu})E_{\nu}(\vb{x})-\frac{Q^2}{2\hbar}\int[\mathrm{d}\vb{k}]f_0\pdv{g_{\nu\lambda}}{k_{\mu}}e_{\nu\lambda}\nonumber\\
        &\hspace{7cm} -\frac{\tau Q^3}{\hbar^2}\int[\mathrm{d}\vb{k}]\pdv{f_0}{k_{\nu}}\Omega_{\mu\gamma}E_{\nu}E_{\gamma}-\frac{\tau Q^3}{2\hbar^2}\int[\mathrm{d}\vb{k}]\pdv{f_0}{k_{\gamma}}\pdv{g_{\nu\lambda}}{k_{\mu}}E_{\gamma}e_{\nu\lambda}.\label{eq:current_general}
    \ee
\end{widetext}
Here, $\qty[\mathrm{d}\vb{k}] = \frac{\mathrm{d}\vb{k}}{(2\pi)^d}$ with the dimension of the system $d$ and the integration is over the whole Brillouin zone. The first three terms in Eq.~\eqref{eq:current_general} have the conventional current caused by the application of the electric field. 

Magnetization and the current density are related by the Maxwell equations:
\be
    \vb{j}&=\curl\vb{H}-\pdv{\vb{D}}{t},
\ee
where
\be
    \vb{H}&=\frac{1}{\mu_0}\vb{B}-\vb{M},\\
    \vb{D}&=\varepsilon_0\vb{E}+\vb{P}.
\ee
Here, we only consider the case where the external magnetic field is zero $\vb{B}=\vb{0}$ and the system is in a steady state $\pdv{\vb{D}}{t}=0$. Thus,
\be
    \vb{j}&=-\curl{\vb{M}}.
\ee
We focus on a magnetization that is induced as a second-order response to the electric field of light and assume that
\be
    M_{\mu}&=a_{\mu\alpha\beta}E_{\alpha}(\vb{x})E_{\beta}(\vb{x}).
\ee
Then, the current can be calculated as
\be
    j_a &= -\varepsilon_{abc}\partial_{b}M_{c}\nonumber\\
    &= -\varepsilon_{abc}\partial_{b}\qty(a_{c\alpha\beta}E_{\alpha}(\vb{x})E_{\beta}(\vb{x}))\nonumber\\
    &=-2\varepsilon_{abc}a_{c\alpha\beta}E_{\alpha}(\vb{x})\partial_bE_{\beta}(\vb{x})\nonumber\\
    &\approx -2\varepsilon_{abc}a_{c\alpha\beta}E_{\alpha}(\vb{x})e_{\beta b}
\ee
By comparing this expression with Eq.~\eqref{eq:current_general}, we find that the last term of Eq.~\eqref{eq:current_general} is responsible for the current caused by the magnetization. We get
\be
    \varepsilon_{\mu\lambda c}a_{c\gamma\nu}&=\frac{\tau Q^3}{4\hbar^2}\int[\mathrm{d}\vb{k}]\pdv{f_0}{k_{\gamma}}\pdv{g_{\nu\lambda}}{k_{\mu}}E_{\gamma}e_{\nu\lambda}.
\ee
Thus, we get a formula for the coefficient
\be
    a_{\mu\alpha\beta}&=\frac{\tau Q^3}{\hbar^2}\varepsilon_{\mu\nu\delta}\int[\mathrm{d}\vb{k}]\pdv{f_0}{k_{\alpha}}\pdv{g_{\delta\beta}}{k_{\nu}}.\label{eq:coeff_metric}.
\ee
From quantum metric, Christoffel symbol can be defined as
\be
    \Gamma_{\alpha\beta\gamma}&\coloneqq\frac{1}{2}\qty(\partial_{k_{\gamma}}g_{\alpha\beta}+\partial_{k_{\beta}}g_{\alpha\gamma}-\partial_{k_{\alpha}}g_{\beta\gamma}).
\ee
Then, it is easy to see
\be
    \partial_{k_{\gamma}}g_{\alpha\beta}&=\Gamma_{\alpha\beta\gamma}+\Gamma_{\beta\alpha\gamma}.
\ee
Using this relation, the coefficient can also be written as
\be
    a_{\mu\alpha\beta}&=\frac{\tau Q^3}{\hbar^2}\varepsilon_{\mu\nu\delta}\int[\mathrm{d}\vb{k}]\pdv{f_0}{k_{\alpha}}\qty(\Gamma_{\delta\beta\nu}+\Gamma_{\beta\delta\nu}).\label{eq:coeff_chris}
\ee

\section{Model Calculation}
Two-dimensional massive Dirac Hamiltonian
\be
    H(k_x,k_y)&=k_x\sigma_x+k_y\sigma_y+M\sigma_z
\ee



\bibliography{IFE.bib}

\end{document}